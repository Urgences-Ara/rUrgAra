\usepackage{titling}
\usepackage{titlesec}
\usepackage{amssymb}
\usepackage{fontspec}
\usepackage{float}
\usepackage{xcolor}
\usepackage{sectsty}
\usepackage{fancyhdr}
\usepackage{graphicx}
\usepackage{xcolor}
\usepackage{lastpage}
\usepackage{calc}
% \usepackage{lscape}%pivote de contenu de la page
\usepackage{pdflscape}%pivote la page entière
\usepackage{tcolorbox}

% Changement titre de la table of content
\renewcommand{\contentsname}{Sommaire}

%base font
\setmainfont{Calibri Light}
%custom font
\newfontfamily\sectionfontcustom{Calibri}

%Customcolor
\definecolor{blueUrgAra}{rgb}{0, 0.36, 0.60}
\definecolor{colSecDemographie}{rgb}{0, 0.69, 0.94}
\definecolor{colSecHospit}{rgb}{0.75, 0, 0}
\definecolor{colSecDuree}{rgb}{0.59, 0.82, 0.31}
\definecolor{colSecGravite}{rgb}{0.79, 0.79, 0.79}
\definecolor{colSecDiag}{rgb}{1, 0.36, 0.60}
\definecolor{colSecSortie}{rgb}{0, 0.44, 0.75}
\definecolor{colSecProvenance}{rgb}{1, 0.60, 0.80}
\definecolor{colSecQualite}{rgb}{1, 0.85, 0.40}

%format titre section
\sectionfont{\color{blueUrgAra}\bfseries\LARGE\sectionfontcustom}
\subsectionfont{\color{white}\bfseries\Large\sectionfontcustom}

%définition commandes
\AtBeginDocument{\let\maketitle\relax}%Supprime la page de garde
\newcommand{\blandscape}{\begin{landscape}}%Begin landscape
\newcommand{\elandscape}{\end{landscape}}%End landscape

%création d'une section invisibles mais présente dans la toc
\newcommand\invisiblesection[1]{%section invisible sur la page, visible pour le sommaire
  \refstepcounter{section}%
  \addcontentsline{toc}{section}{#1}%
  \sectionmark{#1}}

%création du header section
\footskip=56pt
\newcommand{\makesubsection}[3]{%création d'un titre dans un bandeau {titre}{couleur}{icon}
  \fboxsep0pt
  \hspace{-1cm}
  \colorbox{#2}{\hspace{1cm}\begin{minipage}[t][1cm][t]{20cm}
    \vspace{-0.5cm}\hspace{1.1cm}\subsection{#1}
    {\Large
      \vspace{-1.7\baselineskip} % move up
      \hfill
      \includegraphics[height=11mm]{#3}\hspace{0.5cm}
      \vspace{1.2\baselineskip}
      }
    \end{minipage}
  }
}


\newcommand{\makesubsectionnoicon}[2]{%création d'un titre dans un bandeau {titre}{couleur}
  \fboxsep0pt
  \hspace{-1cm}
  \colorbox{#2}{\hspace{1cm}\begin{minipage}[t][1cm][t]{20cm}
    \vspace{-0.5cm}\hspace{1.1cm}\subsection{#1}
    {\Large
      \vspace{-1.7\baselineskip} % move up
      \hspace{0.5cm}
      \vspace{1.2\baselineskip}
      }
    \end{minipage}
  }
}

%création du footer
\def\companylogo{\includegraphics[width=\linewidth,height=70pt,keepaspectratio]{../img/logo_urgara_minimal.jpg}}
\fancypagestyle{companypagestyle}
{
    \fancyhf{}
    \fancyhead{} % clear all header fields
    \renewcommand{\headrulewidth}{0pt} % no line in header area
    \fancyhfoffset[L]{0cm\relax}
    \fancyhfoffset[R]{0cm\relax}
    \fancyfoot[L]
    {
        \parbox[b]{\dimexpr\linewidth-2.2cm\relax}
        {
            Page \thepage\ \hfill Subtitle report \\
            {\color{blueUrgAra}\rule{\dimexpr\linewidth\relax}{0.4pt}}\\
            \vspace{-0.4mm}www.urgences-ara.fr
        }
    }
    \fancyfoot[R]
    {
        \parbox[b]{2cm}{\companylogo}%
    }
}

\pagestyle{companypagestyle}



\newcommand{\avertissementBox}[1]{
  \begin{tcolorbox}[
  colback=blueUrgAra!5!white,
  colframe=blueUrgAra,
  title=Avertissement,
  bottomrule=0.5pt,
  leftrule=0.5pt,
  rightrule=0.5pt]#1
  \end{tcolorbox}}


